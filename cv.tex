%------------------------------------
% Dario Taraborelli
% Typesetting your academic CV in LaTeX
%
% URL: http://nitens.org/taraborelli/cvtex
% DISCLAIMER: This template is provided for free and without any guarantee 
% that it will correctly compile on your system if you have a non-standard  
% configuration.
% Some rights reserved: http://creativecommons.org/licenses/by-sa/3.0/
%------------------------------------

%!TEX TS-program = xelatex
%!TEX encoding = UTF-8 Unicode

\documentclass[11pt, a4paper]{article}
\usepackage{fontspec} 

% DOCUMENT LAYOUT
\usepackage{geometry} 
\geometry{a4paper, textwidth=5.5in, textheight=8.5in, marginparsep=7pt, marginparwidth=.6in}
\setlength\parindent{0in}

% FONTS
\usepackage{xunicode}
\usepackage{xltxtra}
\defaultfontfeatures{Mapping=tex-text} % converts LaTeX specials (``quotes'' --- dashes etc.) to unicode
\setromanfont [Ligatures={Common}, Numbers={OldStyle}]{Hoefler Text}
\setmonofont[Scale=0.8]{Monaco} 
\setsansfont[Scale=0.9]{Optima Regular} 
% ---- CUSTOM AMPERSAND
\newcommand{\amper}{{\fontspec[Scale=.95]{Hoefler Text}\selectfont\itshape\&}}
% ---- MARGIN YEARS
\usepackage{marginnote}
\newcommand{\years}[1]{\marginnote{\scriptsize #1}}
\renewcommand*{\raggedleftmarginnote}{}
\setlength{\marginparsep}{7pt}
\reversemarginpar

% HEADINGS
\usepackage{sectsty} 
\usepackage[normalem]{ulem} 
\sectionfont{\sffamily\mdseries\large\underline} 
\subsectionfont{\rmfamily\mdseries\scshape\normalsize} 
\subsubsectionfont{\rmfamily\bfseries\upshape\normalsize} 

% Japanese input
\usepackage{zxjatype}


% PDF SETUP
% ---- FILL IN HERE THE DOC TITLE AND AUTHOR
\usepackage[xetex, bookmarks, colorlinks, breaklinks, pdftitle={cv_leonardo_pacheco},pdfauthor={Leonardo Pacheco}]{hyperref}
\hypersetup{linkcolor=blue,citecolor=blue,filecolor=black,urlcolor=blue} 

% DOCUMENT
\begin{document}
\textsf{\LARGE Leonardo Kawakami Pacheco}\\[1cm]
Mathematical Institute, Tohoku University\\
Sendai, Miyagi, Japan\\[.2cm]
email: \href{mailto:leonardovpacheco@gmail.com}{leonardovpacheco@gmail.com}\\
site: \href{https://www.leonardopacheco.xyz}{https://www.leonardopacheco.xyz}\\ [.2cm]
% \vfill
Born:  October 19, 1994\\

%%\hrule

\section*{Current Position}
\years{2023--} \textsc{TU Wien}\\
Project Assistant\\

\section*{Education}
\years{2020--2023} PhD in Mathematics at \textsc{Tohoku University}\\
Thesis: Exploring the difference hierarchies on $\mu$-calculus and arithmetic --- from the point of view of Gale--Stewart games \\
Advisor: Keita Yokoyama \\

\years{2018--2020} Master in Mathematics at \textsc{Tohoku University}\\
Thesis: On the weak hierarchy of $\mu$-calculus\\
Advisor: Kazuyuki Tanaka \\

\years{2014--2017} Undergraduate in Mathematics at \textsc{Universidade Federal de Goiás}, Brazil \\

\years{2012--2013} Undergraduate in Computer Science (incomplete) at \textsc{Universidade Federal de Goiás}, Brazil

\section*{Complementary Education}
\years{2019} IMS Graduate Summer School in Logic at NUS, Singapore \\
\years{2016} IMS Graduate Summer School in Logic at NUS, Singapore

\section*{Awards}
\years{2023} Kawai Doctoral Thesis Award from \textsc{Tohoku University}
% http://kawai-zaidan.or.jp/prizes.html


\newpage

\section*{Teaching}
\years{2021} Teaching Assistant --- 数学基礎論特論 (Mathematical Logic --- Graduate)\\
\years{2021} Teaching Assistant --- 計算機数学A  (Computability A)\\
\years{2022} Teaching Assistant --- 計算機数学B  (Computability B)\\

% \hrule

\section*{Publications}
\noindent L. Pacheco, W. Li, K. Tanaka, \emph{On one-variable fragments of modal $\mu$-calculus}, Proceedings of CTFM 2019, World Scientific Publications (2022), 17--45. \\

L. Pacheco, K. Tanaka, \emph{On the degrees of ignorance: via epistemic logic and mu-calculus}, Proceedings of SOCREAL 2022: 6th International Workshop on Philosophy and Logic of Social Reality (2022), 74–78. \\

L. Pacheco, K. Tanaka, \emph{The alternation hierarchy of the mu-calculus over weakly transitive frames} , Proceedings of WoLLIC 2022 – 28th Workshop on Logic, Language, Information and Computation, Lecture Notes in Computer Science, vol 12468, 207--220. \\

L. Pacheco, \emph{Recent Results on Reflection Principles in Second-Order Arithmetic}, RIMS Kôkyûroku No.2228, 73--77. \\

L. Pacheco, K. Yokoyama, \emph{Determinacy and reflection principles in second-order arithmetic}, preprint, \href{https://arxiv.org/abs/2209.04082}{arXiv:2209.04082}.

\section*{Talks}

\years{2023} \emph{Fixed-points in epistemic logic} at \href{https://www.mathsoc.jp/en/meeting/chuo23mar/index.html}{MSJ Spring Meeting 2023}. \\

\years{2022} \emph{The $\mu$-calculus collapses to modal logic over frames of $\mathsf{IS5}$} at \href{https://sites.google.com/view/proof-theory-2022}{Proof Theory Symposium 2022}. \\

\years{2022} \emph{The alternation hierarchy of the $\mu$-calculus over weakly transitive frames} at \href{https://wollic2022.github.io}{28th Workshop on Logic, Language, Information and Computation}. \\

\years{2022} \emph{Determinacy and reflection principles in second-order arithmetic} at \href{https://wrmp2022.sciencesconf.org}{Workshop on Reverse Mathematics and its Philosophy}, joint presentation with Keita Yokoyama. \\

\years{2022} \emph{Determinacy and reflection principles in second-order arithmetic} at \href{https://sites.google.com/view/jp-ru-logic2022/}{The second Japan-Russia workshop on effective descriptive set theory, computable analysis and automata}. \\

\years{2022} \emph{On the degrees of ignorance: via epistemic logic and $\mu$-Calculus} at \href{http://www.asahi-net.or.jp/~yt6t-ymd/sr22/index.html}{SOCREAL 2022}. \\

\newpage

\years{2021} \emph{Sequences of $\beta_k$-models and reflection in second-order arithmetic} at \href{https://www.kyorin-u.ac.jp/univ/user/health/proof_theory/2021/}{Theory and Applications of Proof and Computation --- RIMS共同研究(公開系)}. \\

\years{2021} \emph{On the $\mu$-calculus between $\mathsf{S4}$ and $\mathsf{S5}$} at \href{https://sites.google.com/unal.edu.co/i-enclogbracol/home?authuser=0}{$1^\mathrm{0}$ Enc(ue-o)ntro de Logica Brasil-Col(o-ô)mbia}. \\

\years{2021} \emph{Modal semantics for epistemic logic} at \href{http://www2.kobe-u.ac.jp/~tk/jp/workshop/wakate2021.html}{数学基礎論若手の会 2021}. \\

\years{2020} \emph{$\mu$-calculus and Wadge Degrees} at \href{https://www.kyorin-u.ac.jp/univ/user/health/proof_theory/2020/index.html}{Proof Theory Workshop 2020}.



%\vspace{1cm}
\vfill{}
%\hrulefill

\begin{center}
{\scriptsize  Last updated: \today\\
Access at \href{https://www.leonardopacheco.xyz/cv.pdf}{https://www.leonardopacheco.xyz/cv.pdf}}
\end{center}

\end{document}